\documentclass[a4wide, 11pt]{article}
\usepackage{a4, fullpage}
\setlength{\parskip}{0.3cm}
\setlength{\parindent}{0cm}

% This is the preamble section where you can include extra packages etc.

\begin{document}

\title{Formal BNF Specification}

\author{Alex Rozanski and Thomas Rooney}

\date{\today}         % inserts today's date

\maketitle            % generates the title from the data above

\section*{BNF Grammar} 

Beginning here, we look at the formalised BNF Grammar. This utilises constructs from EBNF and ABNF as explained in the introduction to this specification.

\begin{verbatim}

Declaration     -> Identifier "was a" Type
DeclarationList -> Declaration [{"," Declaration ["too"]} "and" Declaration ["too"]]

Assignment      -> Identifier "became" Expression

BlockUnit       -> "opened" StatementList "closed"

StatementList   -> ""
                 | Statement "."
                 | Statement Split StatementList
               
Statement       -> DeclarationList
                 | Assignment
                 | Expression "said Alice"
                 | Identifier "ate"
                 | Identifier "drank"

Split           -> "."
                 | ","
                 | "then"

Expression      -> Expression + Term
                 | Expression - Term
                 | Term ^ Expression
                 | Term
Term            -> Term * Factor
                 | Term / Factor
                 | Factor
Factor          -> Identifier
                 | Number
                 | "~" (Term | Expression)

Program         -> Function+
Function        -> "The" FunctionType Identifier ArgumentList BlockUnit

FunctionType    -> Type
                 | "looking-glass"
ArgumentList    -> "(" [DeclarationList] ")"

Value           -> Number | Letter

Letter          -> ALPHA

Number          -> ["-"] DIGIT+

Type            -> "number"
                 | "letter"

\end{verbatim}

\end{document}