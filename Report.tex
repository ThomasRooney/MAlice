\documentclass[a4wide, 11pt]{article}
\usepackage{a4, fullpage}
\setlength{\parskip}{0.4cm}
\setlength{\parindent}{0cm}
\usepackage{graphicx}
\usepackage{epstopdf}
\usepackage{float}
\usepackage[margin=1cm]{caption}

\begin{document}

\title{MAlice Report}

\author{Thomas Rooney and Alex Rozanski}

\date{\today}

\maketitle

\section{Our Compiler}

One of the parts of our compiler that we feel is most useful is the syntactic and semantic error annotations. We were heavily inspired by clang's error messages in this regard, and display an ASCII caret character to \'point\' to the locations of such syntactic errors as missing tokens, and the tilde character to \'underline\' larger expressions, such as operands to a mathematical operator.

\subsection{Benchmarks}

In building our compiler we also constructed an autotester. As part of this autotester, we construct benchmarks on the average time to process a file vs the average time it takes the reference compiler to process a file. This can be seen in \'autotest.h\'
\begin{verbatim}
Average Time of Our Compiler / Average Time of Reference Compiler: 10.81822110551374253968
\end{verbatim}

\subsection{Our Extension}

For our extension, we took the principle that a language isn't complete until it can be effectively debugged. Rather than go for a full route of building up a debugging infrastructure however, we decided to build in metadata such that compiled Alice files can be debugged using existing debuggers which support the DWARF debugging format. This includes both gdb and lldb.

The end result of this extension is that a compiled alice file can be debugged in gdb. Breakpoints can be set in the .alice code, variables can be printed and code can be stepped through line by line. To activate, pass the \'-g\' flag to the compile executable.

\begin{verbatim}
(gdb) list
1   
2   The room fibonacci (number x) contained a number
3   opened
4     fib0 was a number of 0 and fib1 was a number of 1 and i was a number.
5     i became 0.
6     eventually (i >= x) because
7       opened
8         temp was a number and temp became fib0.
9         fib0 became fib1.
10        fib1 became temp + fib1.
(gdb) break 5
Breakpoint 1 at 0x40056c: file ../malice_examples/valid/fibonacciIterative.alice, line 5.
(gdb) run
Starting program: /vol/bitbucket/tr111/malice_examples/valid/fibonacciIterative 
Which term in the Fibonacci sequence shall I compute?10

Breakpoint 1, fibonacci ()
    at ../malice_examples/valid/fibonacciIterative.alice:5
5     i became 0.
....
(gdb) next
8         temp was a number and temp became fib0.
(gdb) next
9         fib0 became fib1.
(gdb) print fib0
$2 = 0
(gdb) print fib1
$3 = 1
(gdb) next
10        fib1 became temp + fib1.
(gdb) print fib0
$4 = 1
\end{verbatim}


\section{Design Choices}

For our lexer and parser, we decided to use ANTLR. It is a well-established tool but also has the advantage of a GUI for editing and debugging which makes it a lot better than most other tools. Although we had decided to use C++ for our compiler, we used the C 'target which generated a C lexer and parser from our BNF. We made this decision based on some preliminary research which suggested that the C++ target was less complete.

One disadvantage of using ANTLR was that sometimes the lexer errors produced were not very informative, and were missing vital information such as line or column numbers.

\section{Extensions}





\end{document}
